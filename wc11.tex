\documentclass[a4paper]{exam}

\usepackage{amsfonts,amsmath,amsthm}
\usepackage[a4paper]{geometry}
\usepackage{graphicx}
\usepackage{hyperref}
\graphicspath{{images/}}

\header{CS/MATH 113}{WC11: Mathematical Induction}{Spring 2024}
\footer{}{Page \thepage\ of \numpages}{}
\runningheadrule
\runningfootrule

\printanswers

\qformat{{\large\bf \thequestion. \thequestiontitle}\hfill(\thepoints)}
\boxedpoints

\title{Weekly Challenge 11: Mathematical Induction}
\author{CS/MATH 113 Discrete Mathematics}
\date{Spring 2024}

\begin{document}
\maketitle

\begin{questions}
  \titledquestion{Pyramid Scheme}[10]
  Suppose we have circular coins, a lot of them and all of the same dimension, and we were to make a hexagonal pyramid out of them as follows. The top layer has $1$ coin. The second layer has $7$ coins arranged in a hexagon with side length of two coins (see picture below). The third layer has $19$ coins in the same hexagonal pattern but with side length of three coins, and so on. Prove that a pyramid created in this manner and constituting $n$ layers requires in total $n^3$ coins.
  \begin{figure}[h!]
    \centerline{\includegraphics{layer1}}
    \caption{Top layer as seen from above.}
    \label{layer1}
  \end{figure}
  \begin{figure}[h!]
    \centerline{\includegraphics{layer2.png}}
    \caption{The second layer as seen from above. Note that this layer has $7$ coins and forms a hexagon with side length of two coins.}
    \label{layer2}
  \end{figure}
  \newpage
  \begin{figure}[h!]
    \centerline{\includegraphics{layer3.png}}
    \caption{The third layer with $19$ coins forming a hexagon with side length of three coins.}
    \label{layer3}
  \end{figure}

  Note: All figures are taken from \href{https://www.geogebra.org/m/cnqdjcph}{beckykwarren's geogebra page}.

  \begin{solution}
    % Enter your solution here.
  \end{solution}


\end{questions}

\end{document}
%%% Local Variables:
%%% mode: latex
%%% TeX-master: t
%%% End:
