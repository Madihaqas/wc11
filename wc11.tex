\documentclass[a4paper]{exam}

\usepackage{amsfonts,amsmath,amsthm}
\usepackage[a4paper]{geometry}
\usepackage{graphicx}
\usepackage{hyperref}
\graphicspath{{images/}}

\header{CS/MATH 113}{WC11: Mathematical Induction}{Spring 2024}
\footer{}{Page \thepage\ of \numpages}{}
\runningheadrule
\runningfootrule

\printanswers

\qformat{{\large\bf \thequestion. \thequestiontitle}\hfill(\thepoints)}
\boxedpoints

\title{Weekly Challenge 11: Mathematical Induction}
\author{CS/MATH 113 Discrete Mathematics}
\date{Spring 2024}

\begin{document}
\maketitle

\begin{questions}
  \titledquestion{Pyramid Scheme}[10]
    Suppose we have circular coins, a lot of them and all of the same dimension, and we were to make a hexagonal pyramid out of them as follows. The top layer has $1$ coin. The second layer has $7$ coins arranged in a hexagon with side length of two coins (see picture below). The third layer has $19$ coins in the same hexagonal pattern but with side length of three coins, and so on.

    Use mathematical induction to prove that a pyramid created in this manner and constituting $n$ layers requires in total $n^3$ coins.
  \begin{figure}[h!]
    \centerline{\includegraphics{layer1}}
    \caption{Top layer as seen from above.}
    \label{layer1}
  \end{figure}
  \begin{figure}[h!]
    \centerline{\includegraphics{layer2.png}}
    \caption{The second layer as seen from above. Note that this layer has $7$ coins and forms a hexagon with side length of two coins.}
    \label{layer2}
  \end{figure}
  \newpage
  \begin{figure}[h!]
    \centerline{\includegraphics{layer3.png}}
    \caption{The third layer with $19$ coins forming a hexagon with side length of three coins.}
    \label{layer3}
  \end{figure}

  Note: All figures are taken from \href{https://www.geogebra.org/m/cnqdjcph}{beckykwarren's geogebra page}.

  \begin{solution}
    
    Base Case: \\
    We have to show that $P(n)$ is true for all integers $n$ greater than 1.\\
    For the first case, we have $P(n)=1$, so $P(1)=(1)^3$ is again 1, so the base case holds true.\\
    Inductive Step:\\
    For the inductive hypothesis we assume that for some arbitrary positive integer k, the pyramid with $K$ layers require $K^3$ coins.\\
    We know that the cumulative sum of coins in $K$ layers is $K^3$.\\
    To show that a pyramid with $(K+1)$ layers require $(K+1)^3$ coins.\\
    We know that for $K$ layers, there will be $K^3$ coins, however we do not know how many coins will be there in $K+1$ layers.\\
    To derive the formula for number of cumulative coins, we observe a pattern in the information given to us.\\
    Layer1= 1 coin = 1\\
    Layer2= 1(6) + Layer1 = 7\\
    Layer3= 2(6) + Layer2 = 19\\
    Layer4= 3(6) + Layer3 = 37\\
    Layer5= 4(6) + Layer4 = 61\\
    To find the total number of coins in one layer, we need to find the sum of all the coins in all previous layers.Writing the pattern in a general form to find the total number of coins in n layer:\\
    $L_n$= 6(n-1) + $L_{n-1}$\\ 
     We know that the sum of the natural numbers from 1 upto n can be given by $S_n=\frac{n(n+1)}{2}$ but in this case, we need the formula for $n-1$ integers because n is already known.\\
    $S_{n-1} = \frac{n(n-1)}{2}$\\
    Then $L_n = 6(S_{n-1}) +1$\\
    $6.\frac{n(n-1)}{2} + 1$ which can be simplified to $3n^2-3n+1$. Substituting this formula to find out how many coins are there in $(k+1)^3$ layers\\
    $K^3 + 3(K+1)^2 -3(K+1) +1 = (K+1)^3$\\
    $K^3 + 3(K^2 + 2K + 1) - 3(K + 1) + 1$\\
    $K^3 + 3K^2 + 6K + 3 - 3K - 3 + 1$\\
    $K^3 + 3K^2 + 6K - 3K + 3 - 3 + 1$\\
    $K^3 + 3K^2 + 3K + 1$ which is equivalent to $(K+1)^3$\\
    We have completed the basis step and the inductive step, so by mathematical induction we know that P(n) is true for all positive integers n. That is, we have proven that $P(n)$ = $n^3$ for all positive integers n


  \end{solution}


\end{questions}

\end{document}
%%% Local Variables:
%%% mode: latex
%%% TeX-master: t
%%% End:
